\documentclass[11pt]{article}
\usepackage[spanish]{babel}
%\usepackage{minted}
\usepackage{xcolor}
\usepackage{mdframed}
\usepackage{amsmath}
\usepackage{amssymb}
\usepackage{graphicx}
\usepackage{helvet}
\usepackage{lettrine}
\usepackage{type1cm}
\usepackage{algorithm}
\usepackage{algorithmic}
\usepackage{listings}
\usepackage{mathrsfs}
\usepackage{bm}


\definecolor{wsdred}{HTML}{8E1728}
\definecolor{platino}{HTML}{D9D9D9}
\setlength{\fboxrule}{1pt}

\title{O P T I M I Z A C I Ó N\, \, \textbf{\color{wsdred}III}}
\author{\textit{García García Sacbe, Gutiérrez Salazar Carlos Cuauhtemoc}}
\date{\today}

\begin{document}

\maketitle


\subsection*{Parámetros:}
\begin{itemize}
	\item $S$: Conjunto de sitios de recogida y entrega.
	\item $V$: Conjunto de autobuses.
	\item $d_{ij}$: Distancia entre el nodo $i$ y el nodo $j$; $i,j\in S$.
	\item $q_i$: Cantidad de pasajeros a recoger o entregar en el nodo $i$; $i\in S$.
	\item $\kappa_j$: Capacidad total de los autobuses; $j \in V$ 
\end{itemize}

\subsection*{Variables de decisión:}
\begin{itemize}
	\item $x_{ijk}$: Variable binaria que indica si el vehículo $k$ va desde el nodo $i$ al nodo $j$; $k\in V$, $i,j\in S$.
	\item $y_{ik}$: Variable binaria que indica si el vehículo $k$ visita el nodo $i$; $k\in V$, $i\in S$.
	\item $u_i$: Variable de capacidad acumulada en el sitio $i$; $i\in S$.
\end{itemize}

\subsection*{Función objetivo:}
Minimizar la distancia total recorrida:

Minimizar $\sum_{i,j,k} d_{ij} x_{ijk}$.

\subsection*{Factibilidad:} En el problema los autobuses tienen que pasar por todos los nodos, se tienen que cumplir que la capacidad acumulada de pasajeros no exceda la capacidad del vehículo, tambien se tiene que cumplir que si el vehículo $k$ visita el nodo $i$ y luego el nodo $j$, entonces se recoge en el nodo $i$ antes de la entrega en el nodo $j$, por otra parte se tiene que garantizar que el vehículo $k$ siga un único ciclo en su ruta, por último, 	cada vehículo debe comenzar y terminar en el punto de partida.


\begin{itemize}
	\item Cada nodo debe ser visitado exactamente una vez por algún vehículo:
	$$ \sum_{k} y_{ik} = 1\quad \forall i\in S$$.
	\item Restricciones de capacidad:
	\begin{itemize}
		\item Capacidad inicial en cada sitio es cero:
		$$u_i = 0 \quad \forall i \in S$$
		\item Capacidad acumulada en cada nodo no puede exceder la capacidad del vehículo:
		$$u_i + \sum_{k} q_i y_{ik} \leq \kappa \quad\forall i\in S, k \in V$$.
	\end{itemize}
	\item Restricciones de recogida antes de entrega:
	Si el vehículo $k$ visita el nodo $i$ y luego el nodo $j$, entonces se recoge en el nodo $i$ antes de la entrega en el nodo $j$.
	$$y_{ik} \leq y_{jk} \quad\forall i,j,k$$.
	\item Restricciones de enrutamiento:
	Garantizar que el vehículo $k$ siga un único ciclo en su ruta (rutas de salida igual a las de entrada):
	$$\sum_{i,j} x_{ijk} - \sum_{j} x_{jik} = 0 \quad\forall i,j,k$$.
	\item Restricciones de inicio y fin:
	Cada vehículo debe comenzar y terminar en el depósito:
	$$ \sum_{i} y_{ik} = 2 \quad\forall k\in V $$
	
\end{itemize}

\subsection*{Ejemplo:}
Si tenemos como datos de entrada $S=10$, $V=3$ y $\kappa_j = 50\quad \forall j\in V$, entonces una solución factible sería:\\

\noindent Cargas totales: [44, 44, 44]\\
Vehículo 1: $[(1,3),(3,5),(5,6),(6,2),(2,9),(9,8),(8,7),(7,10),(10,4)$, $(4,1)]$\\
Vehículo 2: $[(1,3),(3,2),(2,5),(5,9),(9,7),(7,10),(10,6),(6,8),(8,4)$, $(4,1)]$\\
Vehículo 3: $[(1,3),(3,4),(4,5),(5,8),(8,7),(7,10),(10,2),(2,6),(6,9)$, $(9,1)]$\\

Se puede ver que es una solución factible ya que
\begin{itemize}
	\item El vehículo inicia y termina en el punto de partida
	\item Se dan una ruta para cada vehículo
	\item Se visita cada uno de los diez sitios por al menos un vehículo
	\item Las cargas de cada vehículo es menor a 50
	\item Todos los vehículos siguen un único ciclo cerrado y no se rompe la continuidad de la ruta, es decir, después de un nodo de la forma $(l,m)$ siempre continua uno de la forma $(m,n)$.
\end{itemize}

\noindent\colorbox{wsdred!20}{$Z = 84$}; Este sería el valor de la función objetivo de la solución, que corresponde a la suma de todas las distancias entre los sitios que recorren todos los distintos autobuses.



\end{document}
